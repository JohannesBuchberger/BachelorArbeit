% ==================================================================
% ============= CUSTOM COMMANDS ====================================
% ==================================================================
\newcommand{\inlinecode}[2]{\colorbox{backcolour}{\raggedright \lstinline[language=#1]$#2$}}

% ===========================================================================================
% define yaml: https://tex.stackexchange.com/questions/152829/how-can-i-highlight-yaml-code-in-a-pretty-way-with-listings/152856#152856
% ===========================================================================================
\newcommand\YAMLcolonstyle{\color{thiblue}\small}
\newcommand\YAMLkeystyle{\color{thiblue}\small}
\newcommand\YAMLvaluestyle{\color{black}\small}

% ===========================================================================================
% Sowas wie eine Sub-Sub-Sub-Section, diese wird nur nicht angezeigt im Inhaltsverzeichnis
% ===========================================================================================
\newcommand{\paragraphandnewline}[1]{\paragraph{#1}\mbox{}}

% ===========================================================================================
% vom \nameref package und \autoref vom hyperref package damit ein gescheid lesbare referenz ensteht 
% ===========================================================================================
\newcommand{\fullref}[1]{\autoref{#1} \nameref{#1}}

% ===========================================================================================
% Erstellt eine Abbildung mit einer Nummer darunter, dass ist die Anzahl der Abbildung. 
% (Bei der 1. steht eine 1, bei der 2. eine 2...
%
%
% Width: /textwidth sollte man immer reinschreiben
% file: link zu der Abbildung die angezeigt werden soll
% label: ref-link für die \ref commands, \fullref commands
% captation: Der Text der optisch neben der oben beschriebenen Nummer steht
% ===========================================================================================

\newcommand{\abbildung}[4]{% [width]{file}{ref-label}{visible-caption}
  \begin{figure}[htbp]
    \centering
    \includegraphics[width=#1]{#2}
    \caption{#4}
    \label{fig:#3}
  \end{figure}
}

\newcommand{\abbildungsvg}[4]{% [width]{file}{ref-label}{visible-caption}
  \begin{figure}[htbp]
    \centering
    \includesvg[width=#1]{#2}
    \caption{#4}
    \label{fig:#3}
  \end{figure}
}
