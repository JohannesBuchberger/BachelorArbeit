% ==================================================================
% ============= Packages & Einstellungen ===========================
% ==================================================================

% ============= Bilder und Bildordnerpfad ==========================
\usepackage{graphicx, subfig}
\graphicspath{ {./img/} }

% ============= Dokumentinformationen ==============================
\usepackage[left= 3 cm,right = 3 cm, top = 3 cm, bottom = 3 cm]{geometry}
\usepackage[onehalfspacing]{setspace}
\usepackage[utf8]{inputenc} % UTF-8 Encoding 
\usepackage[ngerman]{babel}
\usepackage[T1]{fontenc} % Letter encoding
\renewcommand{\familydefault}{\sfdefault}
\usepackage{lmodern} % Schriftart "latex modern"
\usepackage{color} % Farben
\usepackage{eurosym} % euro symbols

% ============= Links ==============================================
\usepackage[
    pdftitle={Empfehlungssysteme},
    pdfsubject={},
    pdfauthor={Johannes Buchberger},
    pdfkeywords={}, 
    %Links nicht einrahmen
    hidelinks
]{hyperref}  % hyperlinks in tableofcontents etc 

\usepackage{nameref}

% ============= mathe stuff ========================================
% zusätzliche Schriftzeichen der American Mathematical Society
\usepackage{mathtools} % mathe tools
\usepackage{amsfonts}
\usepackage{amsmath}
\usepackage[ngerman]{cleveref}

%nicht einrücken nach Absatz
\setlength{\parindent}{0pt}
\usepackage{booktabs}

\usepackage{blindtext}
\usepackage{pdfpages}

% ============= bibtex configs =====================================
\usepackage[backend=biber, style=authoryear, sorting=none]{biblatex}
\DeclareFieldFormat{labelnumberwidth}{}
\setlength{\biblabelsep}{0pt}

\addbibresource{../zusatzverzeichnisse/Literaturverzeichnis.bib}

\renewcommand*{\mkbibparens}[1]{[#1]} % Change "(" to "["

% ============= url configs ========================================
% url's use the same font as the normal text does
\usepackage{url}
\urlstyle{same}
% a url breaks at the end of the line just like everything else does 
% if the url is to long for the page-width
\expandafter\def\expandafter\UrlBreaks\expandafter{\UrlBreaks%  save the current one
  \do\a\do\b\do\c\do\d\do\e\do\f\do\g\do\h\do\i\do\j%
  \do\k\do\l\do\m\do\n\do\o\do\p\do\q\do\r\do\s\do\t%
  \do\u\do\v\do\w\do\x\do\y\do\z\do\A\do\B\do\C\do\D%
  \do\E\do\F\do\G\do\H\do\I\do\J\do\K\do\L\do\M\do\N%
  \do\O\do\P\do\Q\do\R\do\S\do\T\do\U\do\V\do\W\do\X%
  \do\Y\do\Z}

% ============= Abkürzungsverzeichnis ==============================
\usepackage[withpage]{acronym} % acronym, (short names)

% ============= Tabellen ===========================================
\usepackage{tabularx}
\renewcommand{\arraystretch}{1.4}
\usepackage{array,ragged2e}

% for directory tree
\usepackage{forest}

\definecolor{folderbg}{RGB}{124,166,198}
\definecolor{folderborder}{RGB}{110,144,169}

\def\Size{4pt}
\tikzset{
  folder/.pic={
    \filldraw[draw=folderborder,top color=folderbg!50,bottom color=folderbg]
      (-1.05*\Size,0.2\Size+5pt) rectangle ++(.75*\Size,-0.2\Size-5pt);  
    \filldraw[draw=folderborder,top color=folderbg!50,bottom color=folderbg]
      (-1.15*\Size,-\Size) rectangle (1.15*\Size,\Size);
  }
}
% ============= inline code examples ===============================
\usepackage{listings} % put code into your doku, used in \begin
\usepackage{xcolor} % requred for the table

% rotating package allows to rotate text, figures, and tables
\usepackage{rotating} 

 % provides commands for modifying and adjusting the appearance of various elementssuch as text, figures, and tables.
% export option: allows you to use functionalities like the \includegraphics command (used for including images)
\usepackage[export]{adjustbox}

% Allows usage of svg image files
\usepackage{svg}

% change listingverzeichnis name
\renewcommand\lstlistlistingname{Listingverzeichnis}
\renewcommand\lstlistingname{Listing}
 
% set colors for headings
\usepackage{xcolor}
\usepackage{sectsty}
\definecolor{thiblue}{RGB}{1, 63, 107}
\chapterfont{\color{thiblue}}  % sets colour of chapters

\usepackage{float}
 
