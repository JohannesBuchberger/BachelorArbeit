%! Author = jbu
%! Date = 28.09.24

% Preamble
\documentclass[11pt]{article}

% Packages
\usepackage{amsmath}

% Document
\begin{document}

    \section{Abstract}
    ich möchte hiermit meinen Antrag auf ein Bachelorarbeitsthema einreichen. Nach sorgfältiger Überlegung und Beratung mit meinen Betreuern habe ich mich dazu entschlossen, ein Thema zu erforschen, das sich mit der Integration von Prozessinstanzen in eine Vektordatenbank befasst und die Erfolgswahrscheinlichkeit dieser Instanzen unter Berücksichtigung vergangener Prozessinstanzen sowie Metakriterien untersucht.
    \\\\
    \textbf{Abstract:}
    Die vorliegende Arbeit befasst sich mit der Herausforderung, die Erfolgswahrscheinlichkeit von Prozessinstanzen in Vektordatenbanken zu bestimmen. Insbesondere wird untersucht, inwieweit vergangene Prozessinstanzen desselben Prozesses sowie Metakriterien herangezogen werden können, um die Wahrscheinlichkeit des Erfolgs einer neuen Instanz zu prognostizieren. Ein exemplarisches Szenario wird dabei anhand von Flowable und Neo4J betrachtet, wobei Ingenieuren eines Part-Change-Requests Rückmeldung darüber gegeben wird, wie wahrscheinlich es ist, dass der Request erfolgreich ist, ohne Schäden zu verursachen, basierend auf der Ähnlichkeit zur Historie vergangener Prozessinstanzen.
    \\\\
    \textbf{Beschreibung:}
    Die Arbeit wird sich auf die praktische Umsetzung dieser Prognosefähigkeit konzentrieren, wobei für verschiedene Prozesse individuell trainierte Modelle entwickelt werden. Ein Beispiel für diese Anwendung wäre die Änderung einer Bremse, wobei die Verkabelung als Metakriterium betrachtet wird. Durch Analyse vergangener Instanzen, in denen die Verkabelung geändert wurde, kann eine hohe Erfolgswahrscheinlichkeit für ähnliche Änderungen prognostiziert werden. Die Ähnlichkeit zwischen neuen und vergangenen Instanzen wird mithilfe eines linearen Graph-Algorithmus ermittelt, wobei auch Large Language Models (LLMs) zur Identifizierung von Unterschieden in Beschreibungen und anderen Feldern herangezogen werden können.
    \\\\
    \textbf{Ziel:}
    Das Hauptziel dieser Arbeit ist es, ein Verständnis dafür zu entwickeln, wie Vektordatenbanken genutzt werden können, um die Erfolgswahrscheinlichkeit von Prozessinstanzen vorherzusagen. Durch die Integration von vergangenen Instanzen und Metakriterien soll eine praxisnahe Methode entwickelt werden, Ingenieuren und anderen Entscheidungsträgern bei der Bewertung von Prozessänderungen zu unterstützen.
    \\\\
    \textbf{Methodik:}
    Die Methodik umfasst die Entwicklung und Implementierung von Algorithmen zur Bestimmung der Ähnlichkeit zwischen Prozessinstanzen sowie die Erstellung trainierter Modelle für verschiedene Prozesse. Eine umfassende Evaluation der vorgeschlagenen Methode wird durchgeführt, um ihre Zuverlässigkeit und Anwendbarkeit in realen Szenarien zu demonstrieren.
    \\\\
    \textbf{Zeitplan:}

    \begin{enumerate}
        \item Phase 1 (Monat 1-2): Literaturrecherche und Definition der Forschungsziele und -fragen.
        \item Phase 2 (Monat 3-4): Implementierung der Algorithmen und Entwicklung der Trainingsmodelle.
        \item Phase 3 (Monat 5-6): Durchführung von Experimenten und Auswertung der Ergebnisse.
        \item Phase 4 (Monat 7): Verfassen der Bachelorarbeit und Abschluss.
    \end{enumerate}

    Ich bin davon überzeugt, dass diese Arbeit einen bedeutenden Beitrag zum Verständnis der Prognosefähigkeit von Prozessinstanzen in Vektordatenbanken leisten wird. Ich freue mich darauf, diese Forschung zu einem erfolgreichen Abschluss zu bringen und stehe für weitere Diskussionen und Anregungen gerne zur Verfügung.



\end{document}